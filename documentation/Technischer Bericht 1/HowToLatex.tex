\section{HowToLatex}

Einige Beispiele wie Dinge in Latex erreicht werden...


\subsection{Aufzählung}
\begin{itemize}
	\item Erstes Element
	\item Zweites Element
\end{itemize}

Oder mit Zahlen...

\begin{enumerate}
	\item Erstes Element
	\item Zweites Element
\end{enumerate}

\subsection{Formel}
Formel mit Verzeichnis:\\
\begin{align}
    E = m \cdot c^2
\end{align}
\begin{align}
	E = m \frac{4}{7} c^2
\end{align}

Es geht aber auch im Text. Eine Formel wäre $ E=m\cdot c^2 $ und ist sehr berühmt.

\subsection{Tabelle}

\begin{tabular}{|l|c|r|} \hline
Linksbündig & Center & Rechtsbündig \\ \hline 
l & c & r \\ \hline 
\end{tabular}\\

Etwas komplizierter und zentriert:
\begin{center}
\begin{tabular}{|c|l|r|r|} \hline
	\textbf{Klasse} & \textbf{Adressbereich} & \textbf{Anzahl Netze} & \textbf{Interfaces pro Netz} \\ \hline
	A & 1.0.0.0 - 127.255.255.255 & 127 & 16‘777‘214 \\ \hline
	B & 128.0.0.0 – 191.255.255.255 & 16‘384 & 65‘534 \\ \hline
	C & 192.0.0.0 – 223.255.255.255 & 2‘097‘152 & 254 \\ \hline
	D & 224.0.0.0 – 239.255.255.555 & \multicolumn{2}{|c|}{Multicast Adressen} \\ \hline
	E & 240.0.0.0 – 255.255.255.255 & \multicolumn{2}{|c|}{Reserviert für zukünftige Nutzung} \\ \hline
\end{tabular}
\end{center}

\subsection{Bilder}

Einfachs Bild mit fester Breite:\\
\includegraphics[width=3cm]{rec/aBrain-Logo}
\vspace{10mm}

Einfaches Bild mit relativer Breite:\\
\includegraphics[width=0.3\textwidth]{rec/aBrain-Logo}
\vspace{1cm}

\begin{minipage}{0.7\textwidth}
ARP (Address Resolution Protocol) wird benutzt, falls ein Client eine Protokoll-Adresse in eine Hardware-Adresse umsetzten muss und noch keinen entsprechenden Eintrag in der ARP-Tabelle hat. Dieser Client schickt dann einen ARP-Request an alle Teilnehmer mittels Broadcast. Falls ein entsprechender Client sich angesprochen fühlt, schickt dieser eine ARP-Replay Paket zurück.
\end{minipage}
\begin{minipage}{0.3\textwidth}
	\centering
	\includegraphics[width=0.5\textwidth]{rec/aBrain-Logo}
\end{minipage}


